\documentclass{dsekpolicy}

\setadoptedon{VTM Extra 2019}
\setrevisedon{VTM Extra 2023}

\policyför{val}

\begin{document}
\maketitle
\section{Formalia}
\begin{parasection}
  \paraitem{Sammanfattning}
  Denna policyn beskriver hur valprocessen skall gå till både för val gjorda av
  sektionsmötet och styrelsen. Den täcker även hur processen inför valen skall
  gå till.

  \paraitem{Syfte}
  Syftet med denna policy är att förtydliga stadgan och reglementet samt att
  vägleda valberedningen samt andra tillfälliga valberedningar i sitt arbete.

  \paraitem{Historik}
  Utkast färdigställt av Anna Qvil.

  Ursprungligen antagen enligt beslut på VTM extra 2019.

  Reviderad:
  \begin{itemize}
    \item HTM1 2021,
    \item S13 2021,
    \item HTM2 2021 (enl. Policy för Policyer),
    \item VTM 2023, och
    \item VTM-extra 2023 (enl. Policy för styrdokument).
  \end{itemize}

\end{parasection}

\section{Val på sektionsmöten}

\begin{parasection}
  \paraitem{Utlysning}
  Utlysning av de poster som går att söka på sektionsmötet skall anslås senast
  fyra veckor innan mötet. Valberedningens ordförande ansvarar för att utlysningen
  sker.

  Utlysningen ska innehålla en beskrivning av posten, mandatperiod, hur man
  ansöker, sista ansökningsdag, datum för valet samt vem som kan kontaktas för
  ytterligare frågor. Ansökan ska vara öppen minst 5 dagar.

  Senast i samband med utlysning av en kärnpost eller en post som tillsätts av ett
  sektionsmöte ska kravprofil finnas tillgänglig. Vid val där gruppsammansättning
  vägs in ska en kravprofil för gruppen också finnas tillgänglig i samband med
  utlysningen.

  \paraitem{Beredning}
  Val som uträttas av sektionsmötet bereds av valberedningen med undantag för
  Likabehandlingsombud, Studerandeskyddsombud och Världsmästare samt att
  kandidater till valberedningen ej valbereds.

  Trivselrådets kärnposter valbereds genom att Trivselmästare electus sätter ihop
  en valberedning på minst tre medlemmar i sektionen bestående av följande:
  \begin{itemize}
    \item Trivselmästare electus själv,
    \item minst en gammal utskottsmedlem och
    \item minst en medlem i Valberedningen.

  \end{itemize}
  Denna valberedning skall stadfästas på HTM-val. Trivselmästare electus
  ansvarar för att skicka in de funktionärer valberedningen önskar välja in som
  en handling till Höstterminsmöte två. På sektionsmötet skall de redogöra för
  hur processen inför valet gått till.

  \paraitem{Valförfarande}
  Motkandidatur mot valberedningens förslag måste anmälas till Talman senast
  klockan 23:59 två dagar efter valberedningens förslag har offentliggjorts på
  sektionens informationskanaler. Ämnar man motkandidera mot fler poster ska alla
  dessa anmälas.

  Motkandidatur skall anslås omgående samt den föreslagna kandidaten skall
  meddelas. Ansvarig för att meddela kandidaten samt anslå motkandidaten är
  valberedningens ordförande. Endast personer som blivit valberedda för det
  berörda valet får motkandidera.  Fri nominering tillåts endast till de poster
  som valberedningen inte har något förslag till.

  På mötet skall val av styrelsemedlem ha 7 minuter för presentation och 8
  minuter för frågor. Val av annan förtroendepost skall ha 4 minuter för
  presentation och 5 minuter för frågor.
\end{parasection}

\end{document}
