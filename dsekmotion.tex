\documentclass[nopdfbookmarks,a4paper, 11pt, twoside]{article}
\pdfgentounicode=0
\usepackage{dsekcommon}
\usepackage{dsekdokument}
\usepackage[T1]{fontenc}
%\usepackage[utf8]{inputenc}
\usepackage[swedish]{babel}

\newcommand{\MOTE}{HTM1} % Fyll i vilket möte det gäller
\newcommand{\YEAR}{2023} % Fyll i år
\newcommand{\TITLE}{Alfreds fina motion} % Fyll i titel på motionen
\newcommand{\PLACE}{Lund} % Fyll i plats där motionen skrevs, oftast bara "Lund"
\newcommand{\TEXT}{hej hej \dsek} % Fyll i motionens brödtext
\newcommand{\UNDER}{Undertecknad yrkar att sektionsmötet må besluta}
\newcommand{\ATT}[1]{\item #1}
\newcommand{\ATTDESC}[2]{\item #1 \begin{description} \item #2 \end{description}}

\setheader{}{Motion \MOTE - \YEAR}{\PLACE, \today}
\title{Motion: \TITLE}

\begin{document}
\section*{Motion: \TITLE}


\TEXT

\UNDER


\begin{attlista}
	\ATT{detta är en att-sats} % Fyll i att-satsen. Lägg till fler att-satser om det behövs
	\ATTDESC{detta är en att-sats med beskrivning}{detta är en beskrivning}
    \ATT{detta är en till att-sats}
    \ATT{detta är en så pass lång att-sats att den behöver gå över flera rader och då är det bra att den bryts, gör den det är frågan vi alla ställer oss?}
    \ATT{detta är en kort att-sats}
\end{attlista}

\signature{Lund, dag som ovan}{Alfred}{Sektionsmedlem}
\end{document}