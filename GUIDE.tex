% use if creating a proposition
% \documentclass[proposition]{dsekmotion}

% or use this if creating a motion
% \documentclass[motion]{dsekmotion}

% or leave blank to default to motion
\documentclass{dsekdoc}

\usepackage{dsek}
\begin{document}
\settitle{Världens bästa förslag}
\setauthor{Alfred Grip}
\setdate{}
\setmeeting{}
\setshorttitle{Guide till Ger\dsek a}

\section*{Guide till dokumentgeneratorn}
Vad kul att du har hittat hit! 
Det här verktyget har skapats för att det ska bli lättare för gemene sektionsmedlem
att generera stiliga dokument enligt sektionens \LaTeX-mallar.

För få en live-preview av dokumentet så kan du antingen trycka på \textbf{Generera!}-knappen, eller trycka på \texttt{Ctrl/Cmd + S}.

Verktyget har stöd för att skriva i både \LaTeX~och Markdown samtidigt, hur ballt som helst! 
\newline(Tack vare \texttt{pandoc} \url{https://pandoc.org/})

Om du vill bidra till detta verktyg så kan du kika in repot på \url{
    https://github.com/alfredgrip/gerDa
}

Om du istället vill gräva dig ner i sektionens \LaTeX-paket som möjliggör detta så kan du kolla in \url{
    https://github.com/Dsek-LTH/dsekdocs
}

Hoppas du får nytta av det!


\end{document}
