\documentclass[11pt,twoside,a4paper]{article}
\usepackage{dsekbrev}

\begin{document}

\header{SEKTIONSSTYRELSEN}{1999--11--13}

\fromaddress{D-Sektionen inom TLTH \\
  Box 118 \\
  221 00~~Lund}

\toaddress{Bengt Öste \\
  Fiskvägen 66 \\
  123 46~~Norrkøping}

\section*{En kort beskrivning (om det behövs)}

Jag ska i det här dokumentet försöka reda ut och förklara vad
Perl är bra till samt hur det kan användas som ett verktyg för
att förenkla vardagen för programmerare, webmasters och helt
vanliga användare. De exempel jag använder är tagna från
verkligheten.

Så här kan en URL se ut: \url{http://www.dsek.se/}.

Förutom Perl kommer jag även att ta upp ett antal hjälpprogram
som ofta är praktiska att använda tillsammans med Perl, allt för
att skapa en praktisk guide i Perl-användande i stället för en
slavisk genomgång av själva språket.

Viss programmeringsvana kommer att förusättas, dels för att jag
vill begränsa omfånget av texten, och dels för att en väldigt
stor del av de potentiella Perl-användarna redan kan programmera
i något språk. UNIX-vana är knappast nödvändigt, men hjälper
onekligen till.

Dokumentet är inte direkt avsett som referensmaterial; för sådant
finns manualsidor och andra böcker, men jag har sammanställt
några av de viktigaste funktionerna för att man enkelt ska hitta
rätt funktion för en viss uppgift. För den som vill lära sig allt
om Perl finns det också betydligt bättre alternativ.

Perl är ett språk som är inriktat på hantering av text och
automatering av uppgifter. Det lämpar sig därför särskilt väl
till CGI-script för t.ex. dynamisk generering av websidor, men
går utmärkt att använda även för många andra sysslor. Mycket av
det som Perl kan även göras av separata program som
\texttt{grep}, \texttt{sed} och \texttt{awk} om man sätter ihop
dem, men det blir ibland både krångligare och mindre flexibelt.
Naturligtvis skadar det inte att behärska alla verktyg, det är så
man blir effektivast.

Syntaxen är fri och Perl har ett utmärkt stöd för reguljära
uttryck, d.v.s. mönstermatchning av text, vilket man ofta har
väldigt stor nytta av. Den fria syntaxen gör tyvärr Perl-kod
tämligen svårläst för en som inte är invigd. För den som har
lyckan att vara det kan koden göras väldigt kompakt. I Perl finns
det aldrig bara ett sätt att göra något.

Perl lämpar sig dåligt för interaktiva program som kräver
interaktion från användaren. Det finns t.ex. inga funktioner för
att flytta runt markören på skärmen. För detta lämpar sig t.ex. C
bättre, och det finns möjligheter att utnyttja Perl-funktioner i
C-program.

\signature{Med vänliga hälsningar,}{Fnaj Kvatt}{Sektionstomte}

\end{document}
